\documentclass[final,hyperref={pdfpagelabels=false}]{beamer}
\mode<presentation>
  {
  %  \usetheme{Berlin}
  \usetheme{Dreuw}
  }
  \usepackage{times}
  \usepackage{amsmath,amsthm, amssymb, latexsym}
  \boldmath
  \usepackage[english]{babel}
  \usepackage[latin1]{inputenc}
  \usepackage[orientation=portrait,size=a0,scale=1.4,debug]{beamerposter}

  %%%%%%%%%%%%%%%%%%%%%%%%%%%%%%%%%%%%%%%%%%%%%%%%%%%%%%%%%%%%%%%%%%%%%%%%%%%%%%%%%5
  \graphicspath{{figures/}}
  \title[Damegender]{Gender Detection Tool from the Name}
  \author[Arroyo M. \& Gonz�lez B.]{David Arroyo Men�ndez and Jes�s Gonz�lez barahona}
  \institute[GSYC]{Departamento de Sistemas Telem�ticos y Computaci�n, URJC University}
  \date{Mar. 7th, 2019}


  %%%%%%%%%%%%%%%%%%%%%%%%%%%%%%%%%%%%%%%%%%%%%%%%%%%%%%%%%%%%%%%%%%%%%%%%%%%%%%%%%5
  \begin{document}
  \begin{frame}{}
    \vfill
    \begin{block}{\large Why?}
      \centering
      \begin{itemize}
      \item{\large If you want determine gender gap in free software projects or mailing lists}\par
      \item{\large If you don't know the gender about a name in spanish or english}\par
      \item{\large If you want research with statistics about why a name is related with males or females.}\par
      \item{\large If you want use the main solutions in gender detection (genderize, genderapi, namsor, nameapi and gender guesser) from a command.}\par
      \end{itemize}
    \end{block}
    \vfill
    \vfill
    \begin{block}{\large Reducing the gender gap. How? (I)}
      \centering
          {\VERYHuge Create Free Software tools to determine gender in Internet can help to reduce the gender gap in the world}
    \end{block}
    \vfill
    \begin{columns}[t]
      \begin{column}{.48\linewidth}
        \begin{block}{Basic usages}
          \centering
           $$\$ python3 \hspace{1cm}  main.py \hspace{1cm}  David $$
           $$\$ python3 \hspace{1cm}  main.py \hspace{1cm}  Mesa \hspace{1cm} --ml=nltk $$
           $$\$ python3 \hspace{1cm} nameincountries.py \hspace{1cm} David $$
           $$\$ python3 \hspace{1cm} api2gender.py \hspace{1cm} Leticia \hspace{1cm}  \-\-api=genderapi $$
        \end{block}
        \begin{block}{Statistics}
          \begin{itemize}
             \$ python3 \hspace{1cm} infofeatures.py \newline
             \$ python3 \hspace{1cm} confusion.py \newline
             \$ python3 \hspace{1cm} errors.py \hspace{1cm} \-\-csv="files/names/all.csv" \hspace{1cm} \-\-api="genderguesser" \newline
             \$ python3 \hspace{1cm} pca-components.py \hspace{1cm} \-\-csv="files/features\_list.csv" \newline
            \$ python3 \hspace{1cm} pca-features.py \newline
            \$ python3 \hspace{1cm} accuracy.py \hspace{1cm} --api="genderguesser" \hspace{1cm} --measure="recall" \hspace{1cm} --csv=files/names/allnoundefined.csv \par
          \end{itemize}
        \end{block}
      \end{column}
      \begin{column}{.48\linewidth}
        \begin{block}{Easy Installation and Testing}
          \centering
           $$\$ \hspace{1cm} pip3 \hspace{1cm}  install \hspace{1cm}  damegender[all] $$
           $$\$ \hspace{1cm} python3 \hspace{1cm} apikeyadd.py $$
           $$\$ \hspace{1cm} cd \hspace{1cm} src/damegender \hspace{1cm} \& \hspace{1cm} nosetest3 \hspace{1cm} test $$
           $$\$ \hspace{1cm} ./testsbycommands.sh \hspace{1cm} \& \hspace{1cm} ./testsbycommandsextralocal.sh $$
        \end{block}

        \begin{block}{Open Datasets}
          \begin{itemize}
          \item INE.es census names
          \item GenderGuesser
          \item UK and USA births and deatsh
          \item NLTK
          \item Wikidata
          \end{itemize}
        \end{block}
      \end{column}
    \end{columns}
    \begin{block}{\large Reducing the gender gap. How? (II)}
      \centering
          {\VERYHuge Measuring gender gap in software repositories and taking actions about it can help to reduce the gender gap in the world}
    \end{block}
  \end{frame}
\end{document}


%%%%%%%%%%%%%%%%%%%%%%%%%%%%%%%%%%%%%%%%%%%%%%%%%%%%%%%%%%%%%%%%%%%%%%%%%%%%%%%%%%%%%%%%%%%%%%%%%%%%
%%% Local Variables:
%%% mode: latex
%%% TeX-PDF-mode: t
%%% End:
