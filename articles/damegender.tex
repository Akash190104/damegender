%% Submissions for peer-review must enable line-numbering
%% using the lineno option in the \documentclass command.
%%
%% Preprints and camera-ready submissions do not need
%% line numbers, and should have this option removed.
%%
%% Please note that the line numbering option requires
%% version 1.1 or newer of the wlpeerj.cls file, and
%% the corresponding author info requires v1.2

\documentclass[fleqn,10pt,lineno]{wlpeerj} % for journal submissions
% \documentclass[fleqn,10pt]{wlpeerj} % for preprint submissions

\title{Damegender: Writing and Comparing Gender Detection Tools}

\author[1]{David Arroyo Menéndez}
\affil[1]{Universidad Rey Juan Carlos}

\keywords{Gender gap, Gender detection tools, Software repositories}
\begin{abstract}
The gender detection from a name is useful to make gender studies from
social networks, mailing lists, software repositories, papers,
etc. These studies are required to measure the gender gap and to find
solutions to reduce it.

Nowadays, there are several Application Programming Interfaces to
guess gender from a name. This kind of software has the database
based on propietary databases and the software is not free, so some
scientific works are difficult to reproduce

In this paper, we are envisioning how to solve these problems,
offering a tool with a free license to use and compare these
apis. This tool provides Machine Learning to predict strings, that's
useful to guess diminutives or nicknames.
\end{abstract}

\begin{document}

\flushbottom
\maketitle
\thispagestyle{empty}

\section*{1. Introduction}

There are different ways to detect gender from a person name and
perhaps a surname: census, wikipedia, self-references in trust
websites, ... The most common way to detect gender from a name is the
Application Programming Interfaces with a good popularity, for
example, genderapi, namsor, genderize, ...
~\cite{10.7717/peerj-cs.156}

The problems addressed are:
\begin{itemize}[noitemsep]
\item Evaluate quality/price with different commercial solutions.
\item Think about solutions using free licenses.
\item Treatment with names without census, for example, nicknames,
  diminutives, ...
\item Massive gender detection from Internet, for example, mailing
  lists, software repositories, ...
\end{itemize}

In this paper, these problems are being tackling writing a Python
solution for:
\begin{itemize}[noitemsep]
\item To evaluate quality of different solutions applying metrics
  suggested by ~\cite{10.7717/peerj-cs.156}
\item To understand the current technology in detail, I have developed
  a tool guessing gender from a name giving support to Spanish and
  English from the open census.
\item To fix the problem with nicknames and diminutives, we have
  developed a machine learning solution to strings not found in the
  census dataset.
\item To do proof-of-concept tests applying Perceval to detect
  gender in mailing lists and software repositories.
\end{itemize}

In Section 2, we explain the current solutions to the problems. In
Section 3, we present the results evaluating the current Application
Programming Interfaces with our software. In Section 4, we discuss
attempts and problems releasing with a free license a gender detection
from name program. In Section 5, we discuss how to obtain Open Datasets
counting names and gender. In Section 6, we describe our
machine learning solution. In Section 7, we describe general
implementation details. Finally, in Section 8 we summarize our
findings, and describe extensions to the work that we are currently
exploring.

\section*{2. State of Art}

\subsection*{Comparing Commercial Solutions}

A standard commercial Application Programming Interface (API) can
guess the gender for a single name or a list of names (from a CSV file
or an API call). To express geolocalization you can give surnames, a
country ISO code, or a language. Generally, you can give a probability
and a counter associated to a name and gender in a certain population.

~\cite{10.7717/peerj-cs.156} are proposing a good metrics set to
classify these commercial Application Programming Interfaces (features,
measuring errors and success, ...). The features observed are:
Database size (January 2018), Regular data updates, Handles
unstructured full name strings, Handles surnames, Handles non-Latin
alphabets, Implicit geo-localization, Assignment type, Free
parameters, Open source, Application Programming Interface, Monthly
free requests, Monthly subscription cost (100,000 requests/month)).

In the commercial tools is being used different ways to express
probability (confidence, scale, accuracy, precision, recall, ...).

\subsection*{Datasets}

In ~\cite{berners2001semantic} a world was envisioned where public
structured data could be interconnected with software agents to
process these data, perhaps only recovering information, but mixed
with distributed artificial intelligence would give a big jump to the
semantic richness to the web.

~\cite{janssen2012benefits} shows serious profits for the states
adopting Open Data in three categories (1) political and social, (2)
economical, (3) operational and technical. So, Open Data is a
breakthrough towards the Semantic Web.

We can find Open Data about names and gender in census of citizens in
states and commercial solutions. Free software packages such as
~\cite{krawetz2006gender} or ~\cite{loper2002nltk} is providing good
datasets about names and gender. So, Damegender incorporates different
lists of names from free software solutions wrote before (Natural
Language ToolKit, Gender Guesser, ...) and from Open Data census
(United Kingdom, USA, Spain, ...).

Wikidata ~\cite{42240} provides a semantic and open database about
Wikipedia allowing retrieve information with Sparql, such as names and
gender.

~\cite{10.7717/peerj-cs.156} describes different ways to build a
dataset on hand looking for names in papers, scientific websites,
wikipedia, biographies, photos, ...)

\subsection*{Free Software}

Before Damegender, only ~\cite{krawetz2006gender} was competing as
Free Software solution with the main commercial Application
Programming Interfaces about gender detection from the name. The best
contribution is the dataset containing 48528 names with a good
classification by countries.

\subsection*{More software about gender}

In some studies, for example, about Twitter or Github, some people can
choose between different ways to detect gender (not only names). So,
we can find gender detection tools from faces in images
(~\cite{ranjan2017hyperface}), from hand written
(~\cite{liwicki2011automatic}), or from speeches
(~\cite{koppel2002automatically}).

\subsection*{Massive Gender Detection}

There are good studies measuring gender in Internet. Some studies are
about gender gap in general (~\cite{robles2014floss},
~\cite{holman2018gender}, ~\cite{dollar1999gender}), Twitter
(~\cite{burger2011discriminating}, ~\cite{mislove2011understanding})
Stackoverflow (~\cite{vasilescu2012gender}), Wikipedia
(~\cite{antin2011gender}, ~\cite{hill2013wikipedia}), Github
(~\cite{vasilescu2015gender}) ...

\section*{3. Application Programming Interfaces}

\subsection*{Market}

We have reproduced to ~\cite{10.7717/peerj-cs.156} and updated on
27/06/2019 and we are showing the results in
~\ref{table:featuresAndGenderDetectionToolsByName}

\begin{longtable}[]{@{}lllllll@{}}
\toprule
Feature & Gender API & genderguesser & genderize.io & NameAPI & NamSor & Damegender\tabularnewline
\midrule
\endhead
Database size & 431*10^6 & 48.528 & 114*10^6 & 1.428.345 & 4407*10^6  & 57.282\tabularnewline
Regular data updates & yes & no & yes & yes & yes & yes, dev\tabularnewline
Unstructured strings & yes & no & no & yes & no & yes\tabularnewline
Handles surnames & yes & no & no & yes & yes & yes\tabularnewline
Non-Latin alphabets & partially & no & partially & yes & yes & no\tabularnewline
Geo-localization & yes & no & no & yes & yes & no\tabularnewline
Exists locale & yes & yes & yes & yes & yes & yes\tabularnewline
Assingment type & probabilistic & binary & probabilistic & probabilistic & probabilistic & prob\tabularnewline
Free params & total, prob & gender & total, prob & confidence & scale & total, prob\tabularnewline
Guessing with ML & no & no & no & no & no & yes\tabularnewline
Free license & no & yes & no & no & no & yes\tabularnewline
API & yes & no & yes & yes & yes & future\tabularnewline
free requests limited & yes (200) & unlimited & yes (1000) & yes & yes & unlimited\tabularnewline
\bottomrule
\caption{Features and gender detection tools by name}
\label{table:featuresAndGenderDetectionToolsByName}
\end{longtable}

All solutions have increased the database size from
~\cite{10.7717/peerj-cs.156}. NameAPI and GenderAPI is reaching more
features. The tools with a free license have not many features, so for
now that will not be the trend in many situations. Today, one good
solution quality and price is Namsor, which provides unlimited names
through an Application Programming Interface with many features in the
task to detect gender from the name.

\subsection*{Measuring success and errors in gender detection tools from the name}

To guess the sex, we have an true idea (example: female) and we obtain
a result with a method (example: using an api, querying a dataset or
with a machine learning model). The guessed result could be male,
female or perhaps unknown. To remember some vocabulary:

\textbf{True positive} is finding a value guessed as true if the value in
the data source is positive.

\textbf{True negative} is finding a value guessed as true if the the
value in the data source is negative.

\textbf{False positive} is finding a value guessed as false if the the
value in the data source is positive.

\textbf{False negative} is finding a value guessed as false if the the
value in the data source is negative.

In ~\cite{ISO5725}, we can find a vocabulary for measure true, false,
success and errors. We can make a summary in the gender name context
about mathematical concepts:

\textbf{Precision} is about true positives between true positives plus
false positives

\begin{verbatim}
(femalefemale + malemale ) /
(femalefemale + malemale + femalemale)
\end{verbatim}

\textbf{Recall} is about true positives between true positives plus false negatives.

\begin{verbatim}
(femalefemale + malemale ) /
(femalefemale + malemale + malefemale + femaleundefined + maleundefined)
\end{verbatim}

\textbf{Accuray} is about true positives between all.

\begin{verbatim}
(femalefemale + malemale ) /
(femalefemale + malemale + malefemale +
 + femalemale + femaleundefined + maleundefined)
\end{verbatim}

The \textbf{F1 score} is the harmonic mean of precision and recall taking
both metrics into account in the following equation:

\begin{verbatim}
2 * (
(precision * recall) /
(precision + recall))
\end{verbatim}

In Damegender, we are using accuracy.py with the different measures
(precision, recall, accuracy and f1 score) in different apis from an
input. For instance:

\begin{verbatim}
$ python3 accuracy.py --api="damegender" --measure="recall"
  --csv=files/names/allnoundefined.csv
$ python3 accuracy.py --api="damegender" --measure="precision"
  --csv=files/names/allnoundefined.csv
\end{verbatim}

\textbf{Error coded} defines if the true is different than the
guessed. That's divide the number of elements with errors by the
total number of elements:

\begin{verbatim}
(femalemale + malefemale + maleundefined + femaleundefined) /
(malemale + femalemale + malefemale +
 + femalefemale + maleundefined + femaleundefined)
\end{verbatim}

\textbf{Error coded without na} defines if the true is different than
the guessed, but without undefined results. That's divide the number
of elements with undefined errors by the total number of elements

\begin{verbatim}
(maleundefined + femaleundefined) /
(malemale + femalemale + malefemale +
femalefemale + maleundefined + femaleundefined)
\end{verbatim}

\textbf{Error gender bias} allows to understand if the error is bigger
than guessing males than females or viceversa. That's males guessed as
females minus females guessed as males and this number divided by
the total number of elements not guessed as undefined.

\begin{verbatim}
(malefemale - femalemale) /
(malemale + femalemale + malefemale + femalefemale)
\end{verbatim}

\textbf{The weighted error} defines if the true is different than the
guessed, but giving a weight to the guessed as undefined.

\begin{verbatim}
(femalemale + malefemale +
+ w * (maleundefined + femaleundefined)) /
(malemale + femalemale + malefemale + femalefemale +
+ w * (maleundefined + femaleundefined))
\end{verbatim}

In Damegender, we have coded errors.py to implement the different definitions in different apis.

\begin{verbatim}
$ python3 errors.py --api="damegender" --csv=files/names/allnoundefined.csv
Damegender with files/names/allnoundefined.csv has:
+ The error code: 0.2547594323295258
+ The error code without na: 0.2547594323295258
+ The na coded: 0.0
+ The error gender bias: -0.04949809622706819
\end{verbatim}

In the \textbf{confusion matrix} the rows of the datasource element
are true and in the columns the elements are identified as guess.

\begin{verbatim}
[[ 2, 0, 0]
 [ 0, 5, 0]]
\end{verbatim}

It means, I have 2 females true and I've guessed 2 females and I've 5
males true and I've guessed 5 males. I don't have errors in my
classifier.

\begin{verbatim}
[[ 2  1  0]
[ 2 14  0]
\end{verbatim}

It means, I have 2 females true and I've guessed 2 females and I've 14
males true and I've guessed 14 males. 1 female was considered male, 2
males was considered female.

In Damegender, we have coded confusion.py to implement this concept
with the different apis.

\begin{verbatim}
$ python3 confusion.py --csv="files/names/allnoundefined.csv"
\end{verbatim}

\subsection*{Reproducing accuracies and confusion matrix}

~\cite{10.7717/peerj-cs.156} explains different ways to determine
gender from a name by humans and it gives 7000 names applying these
methods. In this dataset the gender is classified as male, female or
unknown. We have used this dataset, but only male and female to these
experiments. We are showing the results in
~\ref{table:DifferentAccuraciesMeasures}

\begin{longtable}[]{@{}lllllll@{}}
  \toprule
  API & Accuracy & Precision & F1score & Recall\tabularnewline
\midrule
\endhead
Genderapi & 0.9687686966482124 & 0.9717050018254838 & 0.9637877964874163 & 1.0\tabularnewline
Genderize & 0.926775 & 0.9761303240374678 & 0.9655113956503119 & 1.0\tabularnewline
Namsor & 0.8672551055728626 & 0.9730097087378641 & 0.9236866359447006 & 1.0\tabularnewline
Nameapi & 0.8301886792452831 & 0.97420272191753 & 0.9054181612233341 & 1.0\tabularnewline
Gender Guesser & 0.7743554248139817 & 0.9848151408450704 & 0.8715900233826968 & 1.0\tabularnewline
Damegender & 0.7452405676704742 & 0.8789548887528067 & 0.8789548887528067 & 1.0\tabularnewline
\bottomrule
\caption{Different accuracies measures}
\label{table:DifferentAccuraciesMeasures}
\end{longtable}

In ~\ref{table:DifferentAccuraciesMeasures} Genderapi and Genderize
are obtaining the best results and the free solutions (Gender Guesser
and Damegender) the worst results. We hope improve Damegender
augmenting languages.

\begin{longtable}[]{@{}lllll@{}}
  \toprule
  APIs        &  gender  &  male   &  female  & undefined \tabularnewline
\midrule
\endhead
Genderapi     & male    & 3589 & 155  &  67 \tabularnewline
              & female  & 211  & 1734 &  23  \tabularnewline
Genderguesser & male    & 3326 &  139  & 346  \tabularnewline
              & female  & 78  & 1686 &  204 \tabularnewline
Genderize     & male    & 3157 & 242  & 412 \tabularnewline
              & female  & 75   & 1742 &  151 \tabularnewline
Nameapi       & male    & 2627 & 674  & 507  \tabularnewline
              & female  & 667  & 1061 &  240  \tabularnewline
Namsor        & male    & 3325 & 139  & 346  \tabularnewline
              & female  & 78   & 1686 &  204  \tabularnewline
Damegender    & male    & 3033 & 778  &   0 \tabularnewline
              & female  & 276  & 1692 &    0 \tabularnewline
\bottomrule
\caption{Confusion matrix tables by APIs}
\label{table:ConfusionMatrixTables}
\end{longtable}

With Damegender has been done a comparison about confusion matrix
tables depending the API (see ~\ref{table:ConfusionMatrixTables}). If
we compare these results with the results obtained in
~\cite{10.7717/peerj-cs.156}, we can understand that the results are
similar.

Genderapi has similar results, but it is being improved the undefined
results. In Genderguesser is we are obtaining different results and it
is strange, because the software has not modified from some years
ago. In Genderize we are obtaining the same results. In Nameapi the
guessed results is changing from male to female with more errors. In
Namsor the results is so similar. Damegender is not guessing undefined
because we predict with machine learning if the string is not in the
database.

The most important tools Namsor, Genderapi and Genderize are improving the
accuracies with respect the previous comparison.

\begin{longtable}[]{@{}lllll@{}}
\toprule
API & error code & error code without na & na coded & error gender bias\tabularnewline
\midrule
\endhead
Genderize & 0.0727 & 0.053 & 0.02 & -0.008\tabularnewline
GenderApi & 0.167 & 0.167 & 0.0 & -0.167\tabularnewline
Namsor & 0.167 & 0.167 & 0.0 & 0.167\tabularnewline
Damegender & 0.255 & 0.255 & 0.0 & -0.049\tabularnewline
Gender Guesser & 0.225 & 0.027 & 0.204 & 0.003\tabularnewline
Nameapi & 0.361 & 0.267 & 0.129 & 0.001 \tabularnewline
\bottomrule
\caption{APIs and Errors}
\label{table:ApisAndErrors}
\end{longtable}

In the table it is possible to observe a high index of errors in
nameapi and a low index of errors in Genderize, GenderApi and Namsor.

\section*{4. Datasets}

We can divide the next options choosing a dataset: (1) a census
published with a free license (open census way), (2) a dataset done by
scientist with a paper in a magazine (scientific way), (3) a dataset
released with a free license in a free software package (free software
way), (4) a dataset retrieved from commercial Application Programming
Interfaces (commercial api way).

\begin{verbatim}
$ python3 main.py David --total="ine"
David gender is male
363559  males for David from INE.es
0 females for David from INE.es
\end{verbatim}

In Damegender, we are including Open Data census about names and
gender, such as INE.es or USA and United Kingdom (births and dies). We
want datasets provided by the software package to increment the speed.

From the user final point of view, the value of using Open Data is
give a good explanation when we are asking about the gender from a
name (number of males and females using a specific name in a country)
versus a probability created by the way explained in
~\cite{10.7717/peerj-cs.156} or similar.

From the scientific point of view, the value of using Open Data is to
allow that the experiment can be reviewed by peers on an automatic and
legal way (using proprietary data the reviewer should request it
separately to make the review).

A second approach is using a dataset from a popular free software
solution. For instance, Natural Language Tool Kit is providing 8000
labeled english names. The classification is male or female. The
problem again is about don't retrieve data with the social science
quality of National Statistics Institutes. Another example is Gender
Guesser a good dataset for international names with different
categories to define the probability. This approach is similar to use
a dataset released with a paper in a journal, the advantage is to
understand and add new names with a solid criteria accepted by the
scientific community.

We are using the census approach as base of truth to distinguish if a
name is male or female in a geographical area. Generally, a name has a
strong weight to determine if it's a male or a female on this way, for
instance, David is registered 363559 times as male and 0 times as
female in Spain National Institute of Statistics.

Many countries don't provide Open Data census about gender and names,
but we envisioned build a Dataset about names and gender free and
universal working from Gender Guesser dataset and Wikidata as
solution. Perhaps, to complete this work we need automate humans
process described in ~\cite{10.7717/peerj-cs.156}.

The last approach is based on to trust on commercial solutions, such
as we trust on search engines to make searches in Internet (black
box).  In Damegender we can download json files from main commercial
Application Programming Interfaces (API) solutions (genderapi,
genderize, namsor, nameapi, ...). One user can build proprietary
datasets on this way using an average weighted by the precision or
accuracy of each Application Programming Interface measured with
Damegender with an open dataset as base of truth.

\section{5. Machine Learning}

These results are experimental, we are improving the choosing of
features and datasets. The datasets used in this experiment was
retrieved from Spain National Institute of Statistics and in Natural
Language ToolKit corpus names (this dataset is about english
names). The features used are: first letter, last letter, a, b, c, d,
e, f, g, h, i, j, k, l, m, n, o, p, q, r, s, t, u, v, w, x, y, z,
vocals, consonants, first letter, first letter vocal, last letter
vocal, last letter consonant, last letter a. The choosing of features
was verified with Principal Component Analysis.

The success with the different algorithms is showed in the next table:

\begin{longtable}[]{@{}lllll@{}}
  \toprule
  Machine Learning Algorithm & Accuracy & Precision & F1score & Recall \tabularnewline
  \midrule
  \endhead
  Tree Decision & 0.745 & 0.879 & 0.879 & 1.0 & \tabularnewline
  Stochastic Gradient Distribution & 0.703 & 0.704 & 0.704 & 1.0 & \tabularnewline
  Gaussian Naive Bayes & 0.703 & 0.704 & 0.704 & 1.0 & \tabularnewline
  Support Vector Machines & 0.698 & 0.994 & 0.994 & 1.0 & \tabularnewline
  Multi Layer Perceptron & 0.677 & 0.82 & 0.756 & 1.0 & \tabularnewline
  Bernoulli Naive Bayes & 0.522 & 0.989 & 0.747 & 1.0 & \tabularnewline
  Multinomial Naive Bayes & 0.406 & 0.99 & 0.576 & 1.0 & \tabularnewline
\bottomrule
\caption{Machine Learning Algorithms and accuracies measures}
\label{table:MLAccuracies}
\end{longtable}

The results in ~\ref{table:MLAccuracies} show that using Tree
Decision for English and Spanish is possible to reach results similar
to another commercial solutions about gender detection tools. Our
classifier is binary (only male and female).

It makes sense expect better results augmenting languages and
countries.

So, it's possible infer that Damegender provides a good solution for
nicknames, diminutives, or similar.

\section*{6. Implementation}

We have chosen Python free software tools with a good scientific
impact. Natural Language Toolkit for Natural Language Processing
~\cite{loper2002nltk}. Scikit for Machine Learning
~\cite{pedregosa2011scikit}. Numpy for Numerical Computation
~\cite{van2011numpy}. Matplotlib to visualize results
~\cite{hunter2007matplotlib}. And Perceval ~\cite{duenas2018perceval}
to retrieve information in mailing lists and repositories.

The current result is a Python package contributed to pip to be used
from the console.

The software is using an oriented to objects design with unit testing
for classes and methods using nose and unit testing for Python
commands using Bash.

A summary of current features in the software are:

\begin{itemize}[noitemsep]
\item To deduce the gender about a name in Spanish or English (current
  status) from open census in local.
\item To decide about males and females in strings using different
  machine learning algorithms.
\item To use the main solutions in gender detection (genderize, genderapi,
namsor, nameapi and gender guesser) from a command.
\item To research about why a name is related to males or females with
  statistics. We provide Python commands about study and compare
  gender solutions with: confusion matrix, accuracies, error
  measures. And to decide about features: statistical feature weight,
  principal component analysis, ...
\item To determine gender gap in free software repositories or mailing
  lists (proof of concept)
\end{itemize}

\section*{7. Conclusions}

The market of gender detection tools is dominated by companies based
on payment services through Application Programming Interfaces with
good results. This market could be modified due to free software
tools and Open Data giving more explicative results for the user.

Although machine learning techniques are not new in this field, we
are giving an approach to guess strings not found in a dataset that
currently is classified as unknown and the humans tendency to think in
gender terms many strings calling it as nicknames or diminutives.

These previous advances in computer science could be giving support to
study the gender gap in repositories and mailing lists. Another future
work is to create a free and universal dataset with support for all
languages and cultures.

\bibliography{bibtex}

\end{document}
